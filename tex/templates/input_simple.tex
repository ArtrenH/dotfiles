%%%%%%%%%%%%%%%%%%%%%%%%%%%%%%%%%
% PACKAGE IMPORTS
%%%%%%%%%%%%%%%%%%%%%%%%%%%%%%%%%


\usepackage[tmargin=2cm,rmargin=1in,lmargin=1in,margin=0.85in,bmargin=2cm,footskip=.2in]{geometry}
\usepackage{amsmath,amsfonts,amsthm,amssymb,mathtools}
\usepackage[varbb]{newpxmath}
\usepackage{xfrac}
\usepackage[makeroom]{cancel}
\usepackage{mathtools}
\usepackage{bookmark}
\usepackage{enumitem}
\usepackage{hyperref,theoremref}
\hypersetup{
	pdftitle={Assignment},
	colorlinks=true, linkcolor=doc!90,
	bookmarksnumbered=true,
	bookmarksopen=true
}
\usepackage[most,many,breakable]{tcolorbox}
\usepackage{xcolor}
\usepackage{varwidth}
\usepackage{varwidth}
\usepackage{etoolbox}
%\usepackage{authblk}
\usepackage{nameref}
\usepackage{multicol,array}
\usepackage[ruled, vlined, linesnumbered, german]{algorithm2e}
\usepackage{comment} % enables the use of multi-line comments (\ifx \fi) 
\usepackage{import}
\usepackage{xifthen}
\usepackage{pdfpages}
\usepackage{transparent}

\usepackage{tikzsymbols}
%\renewcommand\qedsymbol{$\Laughey$}


%%%%%%%%%%%%%%%%%%%%%%%%%%%%%%
% SELF MADE COLORS
%%%%%%%%%%%%%%%%%%%%%%%%%%%%%%


\definecolor{myg}{RGB}{56, 140, 70}
\definecolor{myb}{RGB}{45, 111, 177}
\definecolor{myr}{RGB}{199, 68, 64}
\definecolor{mytheorembg}{HTML}{F2F2F9}
\definecolor{mytheoremfr}{HTML}{00007B}
\definecolor{mylemmabg}{HTML}{E3F2FD}
\definecolor{mylemmafr}{HTML}{1E96FC}
\definecolor{mytheobg}{HTML}{FFFAF8}
\definecolor{mypropbg}{HTML}{f2fbfc}
\definecolor{mypropfr}{HTML}{191971}
\definecolor{myexamplebg}{HTML}{F2FBF8}
\definecolor{myexamplefr}{HTML}{88D6D1}
\definecolor{myexampleti}{HTML}{2A7F7F}
\definecolor{mydefinitbg}{HTML}{E5E5FF}
\definecolor{mydefinitfr}{HTML}{3F3FA3}
\definecolor{notesgreen}{RGB}{0,162,0}
\definecolor{myp}{RGB}{197, 92, 212}
\definecolor{mygr}{HTML}{2C3338}
\definecolor{myred}{RGB}{127,0,0}
\definecolor{myyellow}{RGB}{169,121,69}
\definecolor{myexercisebg}{HTML}{F2FBF8}
\definecolor{myexercisefg}{HTML}{88D6D1}


%%%%%%%%%%%%%%%%%%%%%%%%%%%%%%
% BOX STYLE DEFINITIONS
%%%%%%%%%%%%%%%%%%%%%%%%%%%%%%

\tcbset{Theorem style/.style={
        enhanced,
        breakable,
        colback = mytheorembg,
        frame hidden,
        boxrule = 0sp,
        borderline west = {2pt}{0pt}{mytheoremfr},
        sharp corners,
        detach title,
        before upper = \tcbtitle\par\smallskip,
        coltitle = mytheoremfr,
        fonttitle = \bfseries\sffamily,
        description font = \mdseries,
        separator sign none,
        segmentation style={solid, mytheoremfr},
    }
}

\tcbset{Theoremcon style/.style={
        enhanced
        ,breakable
        ,colback = mytheorembg
        ,frame hidden
        ,boxrule = 0sp
        ,borderline west = {2pt}{0pt}{mytheoremfr}
        ,sharp corners
        ,description font = \mdseries
        ,separator sign none
    }
}


\tcbset{Corollary style/.style={
        enhanced
        ,breakable
        ,colback = myp!10
        ,frame hidden
        ,boxrule = 0sp
        ,borderline west = {2pt}{0pt}{myp!85!black}
        ,sharp corners
        ,detach title
        ,before upper = \tcbtitle\par\smallskip
        ,coltitle = myp!85!black
        ,fonttitle = \bfseries\sffamily
        ,description font = \mdseries
        ,separator sign none
        ,segmentation style={solid, myp!85!black}
    }
}

\tcbset{Lemma style/.style={
        enhanced,
        breakable,
        colback = mylemmabg,
        frame hidden,
        boxrule = 0sp,
        borderline west = {2pt}{0pt}{mylemmafr},
        sharp corners,
        detach title,
        before upper = \tcbtitle\par\smallskip,
        coltitle = mylemmafr,
        fonttitle = \bfseries\sffamily,
        description font = \mdseries,
        separator sign none,
        segmentation style={solid, mylemmafr},
    }
}

\tcbset{Fact style/.style={
        enhanced,
        breakable,
        colback = white,
        frame hidden,
        boxrule = 0sp,
        borderline west = {2pt}{0pt}{black},
        sharp corners,
        detach title,
        before upper = \tcbtitle\par\smallskip,
        coltitle = black,
        fonttitle = \bfseries\sffamily,
        description font = \mdseries,
        separator sign none,
        segmentation style={solid, mylemmafr},
    }
}

\tcbset{Notation style/.style={
    % same as fact style
    }
}

\tcbset{Behauptung style/.style={
        enhanced,
        breakable,
        colback = mytheobg,
        frame hidden,
        boxrule = 0sp,
        borderline west = {2pt}{0pt}{mylemmafr},
        sharp corners,
        detach title,
        before upper = \tcbtitle\par\smallskip,
        coltitle = mylemmafr,
        fonttitle = \bfseries\sffamily,
        description font = \mdseries,
        separator sign none,
        segmentation style={solid, mylemmafr},
    }
}

\tcbset{Prop style/.style={
        enhanced,
        breakable,
        colback = mypropbg,
        frame hidden,
        boxrule = 0sp,
        borderline west = {2pt}{0pt}{mypropfr},
        sharp corners,
        detach title,
        before upper = \tcbtitle\par\smallskip,
        coltitle = mypropfr,
        fonttitle = \bfseries\sffamily,
        description font = \mdseries,
        separator sign none,
        segmentation style={solid, mypropfr},
    }
}

\tcbset{claim style/.style={
        enhanced
        ,breakable
        ,colback = myg!10
        ,frame hidden
        ,boxrule = 0sp
        ,borderline west = {2pt}{0pt}{myg}
        ,sharp corners
        ,detach title
        ,before upper = \tcbtitle\par\smallskip
        ,coltitle = myg!85!black
        ,fonttitle = \bfseries\sffamily
        ,description font = \mdseries
        ,separator sign none
        ,segmentation style={solid, myg!85!black}
    }
}

\tcbset{Exercise style/.style={
        enhanced,
        breakable,
        colback = myexercisebg,
        frame hidden,
        boxrule = 0sp,
        borderline west = {2pt}{0pt}{myexercisefg},
        sharp corners,
        detach title,
        before upper = \tcbtitle\par\smallskip,
        coltitle = myexercisefg,
        fonttitle = \bfseries\sffamily,
        description font = \mdseries,
        separator sign none,
        segmentation style={solid, myexercisefg},
    }
}

\tcbset{Example style/.style={
        colback = myexamplebg
        ,colframe = myexamplefr
        ,coltitle = myexampleti
        ,breakable
        ,boxrule = 1pt
        ,sharp corners
        ,detach title
        ,before upper=\tcbtitle\par\smallskip
        ,fonttitle = \bfseries
        ,description font = \mdseries
        ,separator sign none
        ,description delimiters parenthesis
    }
}

\tcbset{Definition style/.style={
        enhanced,
        colbacktitle=red!75!black,
        before skip=2mm,after skip=2mm, colback=red!5,colframe=red!80!black,boxrule=0.5mm,
        breakable,
        attach boxed title to top left={xshift=1cm,yshift*=1mm-\tcboxedtitleheight}, varwidth boxed title*=-3cm,
        boxed title style={frame code={
                        \path[fill=tcbcolback]
                        ([yshift=-1mm,xshift=-1mm]frame.north west)
                        arc[start angle=0,end angle=180,radius=1mm]
                        ([yshift=-1mm,xshift=1mm]frame.north east)
                        arc[start angle=180,end angle=0,radius=1mm];
                        \path[left color=tcbcolback!60!black,right color=tcbcolback!60!black,
                            middle color=tcbcolback!80!black]
                        ([xshift=-2mm]frame.north west) -- ([xshift=2mm]frame.north east)
                        [rounded corners=1mm]-- ([xshift=1mm,yshift=-1mm]frame.north east)
                        -- (frame.south east) -- (frame.south west)
                        -- ([xshift=-1mm,yshift=-1mm]frame.north west)
                        [sharp corners]-- cycle;
                    },interior engine=empty,
            },
        fonttitle=\bfseries,
    }
}


\tcbset{Note style/.style={
        enhanced jigsaw,
        colback=gray!20!white,%
        colframe=gray!80!black,
        size=small,
        boxrule=1pt,
        title=\textbf{Bemerkung:},
        halign title=flush center,
        coltitle=black,
        breakable,
        drop shadow=black!50!white,
        attach boxed title to top left={xshift=1cm,yshift=-\tcboxedtitleheight/2,yshifttext=-\tcboxedtitleheight/2},
        minipage boxed title=3cm,
        boxed title style={%
                colback=white,
                size=fbox,
                boxrule=1pt,
                boxsep=2pt,
                drop shadow=black!50!white,
                underlay={%
                        \coordinate (dotA) at ($(interior.west) + (-0.5pt,0)$);
                        \coordinate (dotB) at ($(interior.east) + (0.5pt,0)$);
                        \begin{scope}[gray!80!black]
                            \fill (dotA) circle (2pt);
                            \fill (dotB) circle (2pt);
                        \end{scope}
                    },
            },
            before upper = \par\smallskip\hspace{7.5pt}
        #1
    }
}


\tcbuselibrary{theorems,skins,hooks}


% THEOREM BOX
\newcounter{thm} \numberwithin{thm}{chapter}
\newtcbtheorem[number within=chapter, use counter=thm]{Theorem}{Theorem}{Theorem style}{th}
\newtcolorbox{Theoremcon}{Theoremcon style}

% Corollery
\newcounter{cor} \numberwithin{cor}{chapter}
\newtcbtheorem[number within=chapter, use counter=cor]{Corollary}{Korollar}{Corollary style}{th}

% LEMMA
\newcounter{lemma} \numberwithin{lemma}{chapter}
\newtcbtheorem[number within=chapter, use counter=lemma]{mlemma}{Lemma}{Lemma style}{th}

% FAKT
\newcounter{fact} \numberwithin{fact}{chapter}
\newtcbtheorem[number within=chapter, use counter=fact]{fact}{Fakt}{Fact style}{th}

% NOTATION
\newcounter{nota} \numberwithin{nota}{chapter}
\newtcbtheorem[number within=chapter, use counter=nota]{notation}{Notation}{Fact style}{th}

% BEHAUPTUNG
\newtcbtheorem[no counter]{Behauptung}{Behauptung}{Behauptung style}{th}

% PROPOSITION
\newcounter{prop} \numberwithin{prop}{chapter}
\newtcbtheorem[number within=chapter, use counter=prop]{Prop}{Proposition}{Prop style}{th}

% CLAIM BOX
\newcounter{clm} \numberwithin{clm}{chapter}
\newtcbtheorem[number within=chapter, use counter=clm]{claim}{Satz}{claim style}{th}

% EXERCISE BOX
\newtcbtheorem[number within=section]{Exercise}{Exercise}{Exercise style}{th}

% EXAMPLE BOX
\newcounter{ex} \numberwithin{ex}{chapter}
\newtcbtheorem[number within=chapter, use counter=ex]{Example}{Beispiel}{Example style}{ex}

% DEFINITION BOX
\newcounter{dfn} \numberwithin{dfn}{chapter}
\newtcbtheorem[number within=chapter, use counter=dfn]{Definition}{Definition}{Definition style}{def}

% NOTE BOX
\newcounter{nt} \numberwithin{nt}{chapter}
\usetikzlibrary{arrows,calc,shadows.blur}
\newtcbtheorem[number within=chapter, use counter=nt]{note}{Bemerkung}{Note style, theorem name}{def}




%%%%%%%%%%%%%%%%%%%%%%%%%%%%%%
% SELF MADE COMMANDS
%%%%%%%%%%%%%%%%%%%%%%%%%%%%%%


\newcommand{\dfn}[2]{\begin{Definition}{#1}{}#2\end{Definition}}
\newcommand{\dfnstar}[2]{\begin{Definition*}{#1}{}#2\end{Definition*}}
\newcommand{\clm}[2]{\begin{claim}{#1}{}#2\end{claim}}
\newcommand{\clmstar}[2]{\begin{claim*}{#1}{}#2\end{claim*}}
\newcommand{\lemma}[2]{\begin{mlemma}{#1}{}#2\end{mlemma}}
\newcommand{\nota}[2]{\begin{notation}{#1}{}#2\end{notation}}
\newcommand{\notastar}[2]{\begin{notation*}{#1}{}#2\end{notation*}}
\newcommand{\fac}[2]{\begin{fact}{#1}{}#2\end{fact}}
\newcommand{\cor}[2]{\begin{Corollary}{#1}{}#2\end{Corollary}}
\newcommand{\corstar}[2]{\begin{Corollary*}{#1}{}#2\end{Corollary*}}
\newcommand{\prop}[2]{\begin{Prop}{#1}{}#2\end{Prop}}
\newcommand{\propstar}[2]{\begin{Prop*}{#1}{}#2\end{Prop*}}
\newcommand{\ex}[2]{\begin{Example}{#1}{}#2\end{Example}}
\newcommand{\exstar}[2]{\begin{Example*}{#1}{}#2\end{Example*}}
\newcommand{\nt}[2]{\begin{note}{#1}{}#2\end{note}}
\newcommand{\ntstar}[2]{\begin{note*}{#1}{}#2\end{note*}}
\newcommand{\qs}[2]{\begin{question}{#1}{}#2\end{question}}
\newcommand{\qsstar}[2]{\begin{question*}{#1}{}#2\end{question*}}
\newcommand{\prob}[2]{\begin{problem}{#1}{}#2\end{problem}}
\newcommand{\probstar}[2]{\begin{problem*}{#1}{}#2\end{problem*}}
\newcommand{\solu}[2]{\begin{solution}{#1}{}#2\end{solution}}
\newcommand{\solustar}[2]{\begin{solution*}{#1}{}#2\end{solution*}}

\newcommand{\thm}[2]{\begin{Theorem}{#1}{}#2\end{Theorem}}
\newcommand{\wc}[2]{\begin{wconc}{#1}{}#2\end{wconc}}
\newcommand{\thmcon}[1]{\begin{Theoremcon}{#1}\end{Theoremcon}}
\newcommand{\pf}[2]{\begin{myproof}[#1]#2\end{myproof}}
\newcommand{\behauptung}[2]{\begin{Behauptung}{#1}{}#2\end{Behauptung}}

\newcommand{\lecturedivider}[2]{
  \vspace{1em}
  \noindent
  \makebox[\linewidth]{\color{gray!70}\rule{\linewidth}{0.6pt}} % Horizontal line with color
  \vspace{-1.2em}
  \begin{center}
    \textbf{#1. Vorlesung - #2} % Date centered
  \end{center}
  \vspace{-1.5em}
  \noindent
  \makebox[\linewidth]{\color{gray!70}\rule{\linewidth}{0.6pt}} % Another horizontal line
  \vspace{1em}
}


\newcommand{\basicdivider}[1]{
  \vspace{1em}
  \noindent
  \makebox[\linewidth]{\color{gray!70}\rule{\linewidth}{0.6pt}} % Horizontal line with color
  \vspace{-1.2em}
  \begin{center}
    \textbf{#1} % Date centered
  \end{center}
  \vspace{-1.5em}
  \noindent
  \makebox[\linewidth]{\color{gray!70}\rule{\linewidth}{0.6pt}} % Another horizontal line
  \vspace{1em}
}

\newcommand*\circled[1]{\tikz[baseline=(char.base)]{
		\node[shape=circle,draw,inner sep=1pt] (char) {#1};}}
\newcommand\getcurrentref[1]{%
	\ifnumequal{\value{#1}}{0}
	{??}
	{\the\value{#1}}%
}
\newcommand{\getCurrentSectionNumber}{\getcurrentref{section}}
\newenvironment{myproof}[1][\proofname]{%
	\proof[\bfseries #1: ]%
}{\endproof}

\newcommand{\mclm}[2]{\begin{myclaim}[#1]#2\end{myclaim}}
\newenvironment{myclaim}[1][\claimname]{\proof[\bfseries #1: ]}{}

\newcounter{mylabelcounter}

\makeatletter
\newcommand{\setword}[2]{%
	\phantomsection
	#1\def\@currentlabel{\unexpanded{#1}}\label{#2}%
}
\makeatother


%%%%%%%%%%%%%%%%%%%%%%%%%%%%%%
% RANDOM ZEUG %
%%%%%%%%%%%%%%%%%%%%%%%%%%%%%%


\tikzset{
	symbol/.style={
			draw=none,
			every to/.append style={
					edge node={node [sloped, allow upside down, auto=false]{$#1$}}}
		}
}

\newsavebox\diffdbox
\newcommand{\slantedromand}{{\mathpalette\makesl{d}}}
\newcommand{\makesl}[2]{%
\begingroup
\sbox{\diffdbox}{$\mathsurround=0pt#1\mathrm{#2}$}%
\pdfsave
\pdfsetmatrix{1 0 0.2 1}%
\rlap{\usebox{\diffdbox}}%
\pdfrestore
\hskip\wd\diffdbox
\endgroup
}
\newcommand{\dd}[1][]{\ensuremath{\mathop{}\!\ifstrempty{#1}{%
\slantedromand\@ifnextchar^{\hspace{0.2ex}}{\hspace{0.1ex}}}%
{\slantedromand\hspace{0.2ex}^{#1}}}}
\ProvideDocumentCommand\dv{o m g}{%
  \ensuremath{%
    \IfValueTF{#3}{%
      \IfNoValueTF{#1}{%
        \frac{\dd #2}{\dd #3}%
      }{%
        \frac{\dd^{#1} #2}{\dd #3^{#1}}%
      }%
    }{%
      \IfNoValueTF{#1}{%
        \frac{\dd}{\dd #2}%
      }{%
        \frac{\dd^{#1}}{\dd #2^{#1}}%
      }%
    }%
  }%
}
\providecommand*{\pdv}[3][]{\frac{\partial^{#1}#2}{\partial#3^{#1}}}

% redefinde leq and geq
\let\oldleq\leq
\let\oldgeq\geq
\renewcommand{\leq}{\leqslant}
\renewcommand{\geq}{\geqslant}

% % redefine matrix env to allow for alignment, use r as default
% \renewcommand*\env@matrix[1][r]{\hskip -\arraycolsep
%     \let\@ifnextchar\new@ifnextchar
%     \array{*\c@MaxMatrixCols #1}}


%%%%%%%%%%%%%%%%%%%%%%%%%%%%%%%%%%%%%%%%%%%
% TABLE OF CONTENTS
%%%%%%%%%%%%%%%%%%%%%%%%%%%%%%%%%%%%%%%%%%%

\renewcommand{\chaptername}{Kapitel}
\newcommand{\pagename}{Seite }

\usepackage{tikz}
\definecolor{doc}{RGB}{0,60,110}
\usepackage{titletoc}
\contentsmargin{0cm}
\titlecontents{chapter}[3.7pc]
{\addvspace{30pt}%
	\begin{tikzpicture}[remember picture, overlay]%
		\draw[fill=doc!60,draw=doc!60] (-7,-.1) rectangle (-0.9,.5);%
		\pgftext[left,x=-3.5cm,y=0.2cm]{\color{white}\Large\sc\bfseries \chaptername\ \thecontentslabel}
	\end{tikzpicture}\color{doc!60}\large\sc\bfseries}
{}
{}
{\;\titlerule\;\large\sc\bfseries \pagename\ \thecontentspage
	\begin{tikzpicture}[remember picture, overlay]
		\draw[fill=doc!60,draw=doc!60] (2pt,0) rectangle (4,0.1pt);
	\end{tikzpicture}}%
\titlecontents{section}[3.7pc]
{\addvspace{2pt}}
{\contentslabel[\thecontentslabel]{2pc}}
{}
{\hfill\small \thecontentspage}
[]
\titlecontents*{subsection}[3.7pc]
{\addvspace{-1pt}\small}
{}
{}
{\ --- \small\thecontentspage}
[ \textbullet\ ][]

\makeatletter
\renewcommand{\tableofcontents}{%
	\chapter*{%
	  \vspace*{-20\p@}%
	  \begin{tikzpicture}[remember picture, overlay]
		  \pgftext[right,x=15cm,y=0.2cm]{\color{doc!60}\Huge\sc\bfseries \contentsname}
		  \draw[fill=doc!60,draw=doc!60] (10,-.75) rectangle (20,1);
		  \clip (10,-.75) rectangle (20,1);
		  \pgftext[right,x=15cm,y=0.2cm]{\color{white}\Huge\sc\bfseries \contentsname}
	  \end{tikzpicture}}%
	\@starttoc{toc}}
\makeatother


%%%%%%%%%%%%%%%%%%%%%%%%%%%%%%%%%%%%%%%%%%%
% ALIGN WITH LESS VSPACE %
%%%%%%%%%%%%%%%%%%%%%%%%%%%%%%%%%%%%%%%%%%%

\setlength{\jot}{0pt}


%%%%%%%%%%%%%%%%%%%%%%%%%%%%%%%%%%%%%%%%%%%
% SIMPLE REFERENCE MAKER WITH CORRECT COUNTER %
%%%%%%%%%%%%%%%%%%%%%%%%%%%%%%%%%%%%%%%%%%%


\newcommand{\preplabel}[1]{
    \addtocounter{#1}{-1}
    \refstepcounter{#1}
}








\usepackage{silence}
\WarningFilter{latexfont}{}

\usepackage{fancyhdr}
\pagestyle{fancy}
\setlength{\headheight}{12.8064pt}

\usepackage[ngerman]{babel}
\usepackage[T1]{fontenc}

