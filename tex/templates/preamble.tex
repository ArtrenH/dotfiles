%%%%%%%%%%%%%%%%%%%%%%%%%%%%%%%%%
% PACKAGE IMPORTS und Font
%%%%%%%%%%%%%%%%%%%%%%%%%%%%%%%%%



% Hier Schriftart auswählen
\newcounter{schriftart}

% 5 SF Pro Rounded, 1 Roboto, 0 Standard
\setcounter{schriftart}{1}


\ifnum\value{schriftart}=5
	\setmainfont{SF Pro Display} % Schriftart über Counter ändern
	\setsansfont{SF Pro Display}
	\def\balkenx{10.1}
\else
	\ifnum\value{schriftart}=1
		\newcommand{\myfont}{Roboto} % Schriftart über Counter ändern
		\def\balkenx{10.25}
	\else
		\def\balkenx{10.33}
	\fi
\fi
% \usepackage{unicode-math} % Ist nicht mit vielen der Mathematik-Paketen und meier preamble.tex kompatibel
% \setmathfont{Latin Modern Math}  % Alternativ: XITS Math, Asana Math


%\usepackage{soul}
\usepackage[T1]{fontenc}
\usepackage[ngerman]{babel}
\AtBeginDocument{\shorthandoff{"}}
\usepackage[normalem]{ulem}
\usepackage[a4paper,tmargin=2cm,rmargin=1in,lmargin=1in,margin=0.85in,bmargin=2cm,footskip=.2in]{geometry}
\usepackage{amsmath,amsfonts,amsthm,amssymb,mathtools}
%\usepackage[varbb]{newpxmath}

%\usepackage{mathbbol} % Für richtiges Blackboard Bold, aber Vorsicht, macht alles fett

\usepackage{xfrac}
\usepackage[makeroom]{cancel}
\usepackage{mathtools}
\usepackage[atend]{bookmark}
\usepackage{enumitem}
\setlist[itemize]{topsep=6pt plus 2pt minus 1pt, itemsep=3pt plus 2pt minus 1pt}
\usepackage[dvipsnames, table]{xcolor}
\usepackage[most,many,breakable]{tcolorbox}
\usepackage{varwidth}
\usepackage{etoolbox}
%\usepackage{authblk}
\usepackage{nameref}
\usepackage{multicol,array}
\usepackage{tikz-cd}
\usetikzlibrary{babel}
\usetikzlibrary{shapes.geometric}
\usepackage[ruled, vlined,linesnumbered, german]{algorithm2e} % Verwendet den Counter algocf \numberwithin{algocf}{chapter}
\SetArgSty{textnormal} % Macht, dass der Text in algorithm2e nicht kursiv ist
\usepackage{comment} % enables the use of multi-line comments (\ifx \fi)
\usepackage{import}
\usepackage{xifthen}
\usepackage{pdfpages}
\usepackage{transparent}
\usepackage{pgffor} % Macht irgenwas, dass mein Command gleichzeitig mehrere Indexeinträge anlelgen kann

\usepackage{arydshln}     %Matrix machen
\usepackage{nicematrix}
\usepackage{stmaryrd} % Widerspruchzeichen
\usepackage{float}
\usepackage{esint} % Für fintegrale (heißen vielleicht Mittelwertintegrale)
\usepackage{bm} % Für Mathemodus dicker Text
% \usepackage[utf8]{inputenc} % Brauchte ich nur bei pdftex
\usepackage{csquotes} % für deutsche Texte mit Anführungszeichen
\usepackage{centernot} % Für \centernot\implies
\usepackage{pgfplots} % Für Funktionsgraphen
\pgfplotsset{compat=1.18} % Warnmeldung vermeiden
\usepackage{tkz-graph}
\usepackage{caption}
\usepackage{stackengine}
%\usepackage{eucal} % Andere \mathcal{} Symbole
\usepackage{ocgx2}
\usepackage{media9}
\usepackage{pdfpages}
\usetikzlibrary{decorations.pathreplacing,calc}
\usetikzlibrary{decorations.pathmorphing}
\usepackage{silence}
\WarningFilter{latexfont}{}
\usepackage{fancyhdr}
\pagestyle{fancy}
%\setlength{\headheight}{12.8064pt}
\setlength{\headheight}{14.5pt}
\setlength{\footskip}{25pt} % Wie weit die Seitenzahl vom Textende der Seite nach unten eingerückt wird
\fancypagestyle{plain}{} % Auch bei Kapiteln Kopfzeile
\usepackage{changepage} % Für adjustwidth
\usepackage{adjustbox}
\usepackage{makeidx} % Für Index am Ende
\makeindex
\usepackage{hyperref,theoremref} %Hyperref am Ende includen, damit es keine Probleme gibt
\hypersetup{
	pdftitle={Assignment},
	colorlinks=true, linkcolor=linkfarbe,
	bookmarksnumbered=true,
	bookmarksopen=true
}

\usepackage{zref-abspage,zref-user}
\usepackage{atbegshi}
\usepackage{datenumber}
\usepackage{xstring}
%\usepackage{bbm} % ChaptGPT versuch für N (natürlichen Zahlen) fett









\newcommand\mycommfont[1]{\footnotesize\ttfamily\textcolor{blue}{#1}}
\SetCommentSty{mycommfont}
\newcommand{\incfig}[1]{%
    \def\svgwidth{\columnwidth}
    \import{./figures/}{#1.pdf_tex}
}

\usepackage{tikzsymbols}
%\renewcommand\qedsymbol{$\Laughey$}


%\usepackage{import}
%\usepackage{xifthen}
%\usepackage{pdfpages}
%\usepackage{transparent}



% \usepackage{titlesec}

% \titleformat{\chapter}[hang]
%   {\normalfont\Huge\bfseries}
%   {\thechapter.}
%   {1em}
%   {}[\nopagebreak] % kein Zeilenumbruch nach Kapitel

% \titlespacing*{\chapter}{0pt}{0pt}{0pt} % kein Abstand vor oder nach Kapitel

\usepackage{etoolbox}
\makeatletter
\patchcmd{\chapter}{\if@openright\cleardoublepage\else\clearpage\fi}{}{}{}
\patchcmd{\chapter}{\clearpage}{}{}{}
\makeatother


%%%%%%%%%%%%%%%%%%%%%%%%%%%%%%
% SELF MADE COLORS
%%%%%%%%%%%%%%%%%%%%%%%%%%%%%%


\definecolor{doc}{RGB}{0,60,110}
\definecolor{doc}{HTML}{0B1D51}
\definecolor{docg}{HTML}{00BB00} % hellgrün
\definecolor{doc}{HTML}{064789} % geeignetes blau für Inhaltsverzeichnis

\definecolor{alternativlink}{HTML}{1C6CB3} % Farbe für Hyperlinks
\colorlet{linkfarbe}{doc!80!white} % Farbe für Hyperlinks

\definecolor{myg}{RGB}{56, 140, 70}
\definecolor{myb}{RGB}{45, 111, 177}
\definecolor{myr}{RGB}{199, 68, 64}
\definecolor{mytheorembg}{HTML}{F2F2F9}
\definecolor{mytheoremfr}{HTML}{00007B}
\definecolor{mylemmabg}{HTML}{e3f2fd}
\definecolor{mylemmafr}{HTML}{1e96fc}
\definecolor{mypropbg}{HTML}{d7c8f3}
\definecolor{mypropfr}{HTML}{8b26c3}
\definecolor{myexamplebg}{HTML}{EAFBF8}
\definecolor{myexamplefr}{HTML}{2A7F7F}
\definecolor{myexampleti}{HTML}{2A7F7F}
\definecolor{myexamplebg}{RGB}{24, 183, 221}
\definecolor{myexamplefr}{RGB}{24, 183, 221}
\definecolor{mydefinitbg}{HTML}{E5E5FF}
\definecolor{mydefinitfr}{HTML}{3F3FA3}
\definecolor{notesgreen}{RGB}{0,162,0}
\definecolor{myp}{RGB}{197, 92, 212}
\definecolor{mygr}{HTML}{2C3338}
\definecolor{myred}{RGB}{127,0,0}
\definecolor{myyellow}{RGB}{169,121,69}
\definecolor{myexercisebg}{HTML}{023e7d}
\definecolor{myexercisefr}{HTML}{023e8a}
\definecolor{myclaimbg}{HTML}{99ca3c}
\definecolor{myclaimfr}{HTML}{208b3a}
\definecolor{mynotationbg}{HTML}{fff3b0}
\definecolor{mynotationfr}{HTML}{ffb700}
\definecolor{mynotationfr}{RGB}{227, 173, 15}
\definecolor{myntbg}{HTML}{f8f9fa}
\definecolor{myntfr}{HTML}{B0B2B8}
\definecolor{myfolg}{RGB}{118, 186, 13}



%%%%%%%%%%%%%%%%%%%%%%%%%%%%
% TCOLORBOX SETUPS
%%%%%%%%%%%%%%%%%%%%%%%%%%%%

\setlength{\parindent}{0cm} % Keine scheiß automatische Einrückung

% Der globale Counter zum abhängigen Zählen

\newcounter{globalcounter} %\numberwithin{globalcounter}{chapter} % Definiere einen globalen Zähler für alle
%\counterwithout{globalcounter}{chapter}
\renewcommand{\theglobalcounter}{\arabic{globalcounter}} % Globale Nummerierung ohne Kapitelbindung






%================================
% Aufgabe
%================================

\newcounter{aufg} \numberwithin{aufg}{chapter}

\tcbuselibrary{theorems,skins,hooks}
\tcbset{Exercise style/.style={
	%
	enhanced,
	breakable,
	colback = myexercisebg!15,
	frame hidden,
	boxrule = 0sp,
	borderline west = {3pt}{0pt}{myexercisefr},
	sharp corners,
	detach title,
	before upper = \tcbtitle\par\smallskip,
	coltitle = myexercisefr,
	fonttitle = \bfseries\sffamily,
	description font = \mdseries,
	separator sign none,
	segmentation style={solid, myexercisefr},
	}
}
\newtcbtheorem[number within=chapter, use counter=globalcounter]{Exercise}{Aufgabe}{Exercise style}{aufg}

%================================
% Behauptung
%================================

\tcbuselibrary{theorems,skins,hooks}
\tcbset{Behauptung style/.style={
	%
	enhanced,
	breakable,
	colback = mylemmabg,
	frame hidden,
	boxrule = 0sp,
	borderline west = {3pt}{0pt}{mylemmafr},
	sharp corners,
	detach title,
	before upper = \tcbtitle\par\smallskip,
	coltitle = mylemmafr,
	fonttitle = \bfseries\sffamily,
	description font = \mdseries,
	separator sign none,
	segmentation style={solid, mylemmafr},
	}
}
\newtcbtheorem[no counter]{Behauptung}{Behauptung}{Behauptung style}{th}

%================================
% Beispiel
%================================

\newcounter{ex} \numberwithin{ex}{chapter}
\tcbset{Example style/.style={
	%
	colback = myexamplebg!9,
	breakable,
	enhanced jigsaw,
	colframe = myexamplefr!85!black,
	coltitle = myexamplefr!85!black,
	boxrule = 2pt,
	detach title,
	before upper=\tcbtitle\par\smallskip,
	description font = \mdseries,
	separator sign none,
	description delimiters parenthesis,
	fonttitle = \bfseries\sffamily
	}
}
\newtcbtheorem[number within=chapter, use counter=globalcounter]{Example}{Beispiel}{Example style}{ex}

%================================
% Bemerkung
%================================

\newcounter{nt} \numberwithin{nt}{chapter}

\usetikzlibrary{arrows,calc,shadows.blur}
\tcbuselibrary{skins}
\tcbset{note style/.style={
	enhanced jigsaw,
	colback=myntbg,%
	colframe=myntfr,
	boxrule=1pt,
	%title=\textbf{Bemerkung:-},
	halign title=flush center,
	coltitle=black,
	breakable,
	breakable,
	drop shadow=black!50!white,
	attach boxed title to top left={xshift=1cm,yshift=-\tcboxedtitleheight/2,yshifttext=-\tcboxedtitleheight/2},
	boxed title style={%
			colback=white,
			boxrule=1pt,
			boxsep=2pt,
			drop shadow=black!50!white,
			underlay={%
					\begin{scope}[myntfr]
						\fill ($(interior.west) + (-0.5pt,0)$) circle (2pt);
						\fill ($(interior.east) + (0.5pt,0)$) circle (2pt);
					\end{scope}
				},
		},
		width=\textwidth, % Verwendet den gesamten verfügbaren Platz
	#1
	}
}
\newtcbtheorem[number within=chapter, use counter=globalcounter]{note}{Bemerkung}{note style}{def}

%================================
% Definition
%================================

\newcounter{dfn} \numberwithin{dfn}{chapter}
\tcbset{Definition style/.style={
	enhanced,
    colbacktitle=red!75!black,
	%separator sign={\if\ifnocountertrue \hspace{-4pt}: \else:\fi},
	before skip=2mm,after skip=2mm, colback=red!5,colframe=red!80!black,boxrule=2pt,
	breakable,
    enforce breakable,
	attach boxed title to top left={xshift=1cm,yshift*=1mm-\tcboxedtitleheight}, varwidth boxed title*=-3cm,
	boxed title style={
		frame code={
			\path[fill=tcbcolback]
			([yshift=-1mm,xshift=-1mm]frame.north west)
			arc[start angle=0,end angle=180,radius=1mm]
			([yshift=-1mm,xshift=1mm]frame.north east)
			arc[start angle=180,end angle=0,radius=1mm];
			\path[left color=tcbcolback!70!black,right color=tcbcolback!70!black,middle color=tcbcolback!80!black]
			([xshift=-2mm]frame.north west)--
			([xshift=2mm]frame.north east)
			[rounded corners=1mm]-- ([xshift=1mm,yshift=-1mm]frame.north east)--
			([yshift=-1pt]frame.south east)--
			([yshift=-1pt]frame.south west)--
			([xshift=-1mm,yshift=-1mm]frame.north west)
			[sharp corners]-- cycle;
		},
	interior engine=empty,
	},
	fonttitle = \bfseries\sffamily
	%title={Definition}
	#1
	}
}
\newtcbtheorem[number within=chapter, use counter=globalcounter]{Definition}{Definition}{Definition style}{def}

%================================
% Fakt
%================================

\newcounter{fact} \numberwithin{fact}{chapter}

\tcbuselibrary{theorems,skins,hooks}
\tcbset{fact style/.style={
	%
	enhanced,
	breakable,
	colback = white,
	frame hidden,
	boxrule = 0sp,
	borderline west = {3pt}{0pt}{black},
	sharp corners,
	detach title,
	before upper = \tcbtitle\par\smallskip,
	coltitle = black,
	fonttitle = \bfseries\sffamily,
	description font = \mdseries,
	separator sign none,
	segmentation style={solid, mylemmafr},
	}
}
\newtcbtheorem[number within=chapter, use counter=globalcounter]{fact}{Fakt}{fact style}{th}

%================================
% Folgerung
%================================

\newcounter{folg} \numberwithin{folg}{chapter}

\tcbuselibrary{theorems,skins,hooks}
\tcbset{Folgerung style/.style={%
	enhanced,
	breakable,
	colback = myfolg!10,
	frame hidden,
	boxrule = 0sp,
	borderline west = {3pt}{0pt}{myfolg},
	sharp corners,
	detach title,
	before upper = \tcbtitle\par\smallskip,
	coltitle = myfolg,
	fonttitle = \bfseries\sffamily,
	description font = \mdseries,
	separator sign none,
	segmentation style={solid, myfolg}
	}
}
\newtcbtheorem[number within=chapter, use counter=globalcounter]{Folgerung}{Folgerung}{Folgerung style}{folg}

%================================
% Frage
%================================

\makeatletter

\tcbset{question style/.style={
	enhanced,
	breakable, boxrule = 2pt,
	fonttitle = \sffamily,
	colback=white,
	colframe=myb!80!black,
	attach boxed title to top left={yshift*=-\tcboxedtitleheight},
	%fonttitle=\bfseries,
	%title={#2},
	boxed title size=title,
	boxed title style={%
			sharp corners,
			rounded corners=northwest,
			colback=tcbcolframe,
			boxrule=0pt,
		},
	underlay boxed title={%
			\path[fill=tcbcolframe] (title.south west)--
			(title.south east) to[out=0, in=180] ([xshift=5mm]title.east)--
			(title.center-|frame.east)
			[rounded corners=\kvtcb@arc] |-
			(frame.north) -| cycle;
		},
	#1
	}
}
\newtcbtheorem[number within=chapter, use counter=globalcounter]{question}{Question}{question style}{def}

%================================
% Korollar
%================================

\newcounter{cor} \numberwithin{cor}{chapter}

\tcbuselibrary{theorems,skins,hooks}
\tcbset{Corollary style/.style={
	%
	enhanced,
	breakable,
	colback = myp!10,
	frame hidden,
	boxrule = 0sp,
	borderline west = {3pt}{0pt}{myp!85!black},
	sharp corners,
	detach title,
	before upper = \tcbtitle\par\smallskip,
	coltitle = myp!85!black,
	fonttitle = \bfseries\sffamily,
	description font = \mdseries,
	separator sign none,
	segmentation style={solid, myp!85!black}
	}
}
\newtcbtheorem[number within=chapter, use counter=globalcounter]{Corollary}{Corollary}{Corollary style}{th}

%================================
% Lemma
%================================

\newcounter{lemma} \numberwithin{lemma}{chapter}

\tcbuselibrary{theorems,skins,hooks}
\tcbset{mlemma style/.style={
	%
	enhanced,
	breakable,
	colback = mylemmabg,
	frame hidden,
	boxrule = 0sp,
	borderline west = {3pt}{0pt}{mylemmafr},
	sharp corners,
	detach title,
	before upper = \tcbtitle\par\smallskip,
	coltitle = mylemmafr,
	fonttitle = \bfseries\sffamily,
	description font = \mdseries,
	separator sign none,
	segmentation style={solid, mylemmafr},
	}
}
\newtcbtheorem[number within=chapter, use counter=globalcounter]{mlemma}{Lemma}{mlemma style}{th}

%================================
% Lösung
%================================

\makeatletter
\tcbset{solution style/.style={
	enhanced, boxrule = 2pt,
	breakable,
	fonttitle = \sffamily,
	colback=white,
	colframe=myg!80!black,
	attach boxed title to top left={yshift*=-\tcboxedtitleheight},
	%title=Lösung,
	boxed title size=title,
	boxed title style={%
			sharp corners,
			rounded corners=northwest,
			colback=tcbcolframe,
			boxrule=0pt,
		},
	underlay boxed title={%
			\path[fill=tcbcolframe] (title.south west)--
			(title.south east) to[out=0, in=180] ([xshift=5mm]title.east)--
			(title.center-|frame.east)
			[rounded corners=\kvtcb@arc] |-
			(frame.north) -| cycle;
		},
	}
}
\newtcbtheorem[number within=chapter, use counter=globalcounter]{solution}{Solution}{solution style}{def}

%================================
% Notation
%================================

\newcounter{notation} \numberwithin{notation}{chapter}

\tcbuselibrary{theorems,skins,hooks}
\tcbset{notation style/.style={
	%
	enhanced,
	breakable,
	colback = mynotationbg!50,
	frame hidden,
	boxrule = 0sp,
	borderline west = {3pt}{0pt}{mynotationfr},
	sharp corners,
	detach title,
	before upper = \tcbtitle\par\smallskip,
	coltitle = mynotationfr,
	fonttitle = \bfseries\sffamily,
	description font = \mdseries,
	separator sign none,
	segmentation style={solid, mynotationfr},
	}
}
\newtcbtheorem[number within=chapter, use counter=globalcounter]{notation}{Notation}{notation style}{th}

%================================
% Proposition
%================================

\newcounter{prop} \numberwithin{prop}{chapter}

\tcbuselibrary{theorems,skins,hooks}
\tcbset{Prop style/.style={
	%
	enhanced,
	breakable,
	colback = mypropbg!30,
	frame hidden,
	boxrule = 0sp,
	borderline west = {3pt}{0pt}{mypropfr},
	sharp corners,
	detach title,
	before upper = \tcbtitle\par\smallskip,
	coltitle = mypropfr,
	fonttitle = \bfseries\sffamily,
	description font = \mdseries,
	separator sign none,
	segmentation style={solid, mypropfr},
	}
}
\newtcbtheorem[number within=chapter, use counter=globalcounter]{Prop}{Proposition}{Prop style}{th}


%================================
% Construction
%================================

\newcounter{const} \numberwithin{const}{chapter}

\tcbuselibrary{theorems,skins,hooks}
\newtcbtheorem[number within=chapter, use counter=globalcounter]{Const}{Construction}{Prop style}{th}

%================================
% Satz
%================================

\newcounter{clm} \numberwithin{clm}{chapter}

\tcbuselibrary{theorems,skins,hooks}
\tcbset{claim style/.style={
	%
	enhanced,
	breakable,
	colback = myclaimbg!20,
	frame hidden,
	boxrule = 0sp,
	borderline west = {3pt}{0pt}{myclaimfr},
	sharp corners,
	detach title,
	before upper = \tcbtitle\par\smallskip,
	coltitle = myclaimfr,
	fonttitle = \bfseries\sffamily,
	description font = \mdseries,
	separator sign none,
	segmentation style={solid, myclaimfr}
	}
}
\newtcbtheorem[number within=chapter, use counter=globalcounter]{claim}{Claim}{claim style}{th}

%================================
% Theorem
%================================

\newcounter{thm} \numberwithin{thm}{chapter}

\tcbuselibrary{theorems,skins,hooks}
\tcbset{Theorem style/.style={%
	enhanced,
	breakable,
	colback = mytheorembg,
	frame hidden,
	boxrule = 0sp,
	borderline west = {3pt}{0pt}{mytheoremfr},
	sharp corners,
	detach title,
	before upper = \tcbtitle\par\smallskip,
	coltitle = mytheoremfr,
	fonttitle = \bfseries\sffamily,
	description font = \mdseries,
	separator sign none,
	segmentation style={solid, mytheoremfr},
	}
}
\newtcbtheorem[number within=chapter, use counter=globalcounter]{Theorem}{Theorem}{Theorem style}{th}
\tcbuselibrary{theorems,skins,hooks}
\newtcolorbox{Theoremcon}
{%
	enhanced,
	breakable,
	colback = mytheorembg,
	frame hidden,
	boxrule = 0sp,
	borderline west = {3pt}{0pt}{mytheoremfr},
	sharp corners,
	description font = \mdseries,
	separator sign none
}
\makeatother





%%%%%%%%%%%%%%%%%%%%%%%%%%%%%%
% SELF MADE COMMANDS
%%%%%%%%%%%%%%%%%%%%%%%%%%%%%%


\newcommand{\dfn}[2]{\begin{Definition}{#1}{}#2\end{Definition}}
\newcommand{\clm}[2]{\begin{claim}{#1}{}#2\end{claim}}
\newcommand{\lemma}[2]{\begin{mlemma}{#1}{}#2\end{mlemma}}
\newcommand{\nota}[2]{\begin{notation}{#1}{}#2\end{notation}}
\newcommand{\fac}[2]{\begin{fact}{#1}{}#2\end{fact}}
\newcommand{\cor}[2]{\begin{Corollary}{#1}{}#2\end{Corollary}}
\newcommand{\prop}[2]{\begin{Prop}{#1}{}#2\end{Prop}}
\newcommand{\const}[2]{\begin{Const}{#1}{}#2\end{Const}}
\newcommand{\ex}[2]{\begin{Example}{#1}{}#2\end{Example}}
\newcommand{\nt}[2]{\begin{note}{#1}{}#2\end{note}}
\newcommand{\qs}[2]{\begin{question}{#1}{}#2\end{question}}
\newcommand{\solu}[2]{\begin{solution}{#1}{}#2\end{solution}}
\newcommand{\folg}[2]{\begin{Folgerung}{#1}{}#2\end{Folgerung}}

\newcommand{\aufg}[2]{\begin{Exercise}{#1}{}#2\end{Exercise}}
\newcommand{\thm}[2]{\begin{Theorem}{#1}{}#2\end{Theorem}}
\newcommand{\thmcon}[1]{\begin{Theoremcon}{#1}\end{Theoremcon}}
\newcommand{\pf}[2]{\begin{myproof}[#1]#2\end{myproof}}
\newcommand{\behauptung}[2]{\begin{Behauptung}{#1}{}#2\end{Behauptung}}

% Variante ohne Counter
\newcommand{\dfnstern}[2]{\begin{Definition*}{#1}{}#2\end{Definition*}}
\newcommand{\clmstern}[2]{\begin{claim*}{#1}{}#2\end{claim*}}
\newcommand{\lemmastern}[2]{\begin{mlemma*}{#1}{}#2\end{mlemma*}}
\newcommand{\notastern}[2]{\begin{notation*}{#1}{}#2\end{notation*}}
\newcommand{\facstern}[2]{\begin{fact*}{#1}{}#2\end{fact*}}
\newcommand{\corstern}[2]{\begin{Corollary*}{#1}{}#2\end{Corollary*}}
\newcommand{\propstern}[2]{\begin{Prop*}{#1}{}#2\end{Prop*}}
\newcommand{\conststern}[2]{\begin{Const*}{#1}{}#2\end{Const*}}
\newcommand{\exstern}[2]{\begin{Example*}{#1}{}#2\end{Example*}}
\newcommand{\ntstern}[2]{\begin{note*}{#1}{}#2\end{note*}}
\newcommand{\qsstern}[2]{\begin{question*}{#1}{}#2\end{question*}}
\newcommand{\solustern}[2]{\begin{solution*}{#1}{}#2\end{solution*}}
\newcommand{\folgstern}[2]{\begin{Folgerung*}{#1}{}#2\end{Folgerung*}}

\newcommand{\aufgstern}[2]{\begin{Exercise*}{#1}{}#2\end{Exercise*}}
\newcommand{\thmstern}[2]{\begin{Theorem*}{#1}{}#2\end{Theorem*}}
\newcommand{\thmconstern}[1]{\begin{Theoremcon*}{#1}\end{Theoremcon*}}
\newcommand{\pfstern}[2]{\begin{myproof}[#1]#2\end{myproof}}
\newcommand{\behauptungstern}[2]{\begin{Behauptung*}{#1}{}#2\end{Behauptung*}}

% star instead of stern

\let\dfnstar\dfnstern
\let\clmstar\clmstern
\let\lemmastar\lemmastern
\let\notastar\notastern
\let\facstar\facstern
\let\corstar\corstern
\let\propstar\propstern
\let\conststar\conststern
\let\exstar\exstern
\let\ntstar\ntstern
\let\qsstar\qsstern
\let\solustar\solustern
\let\folgstar\folgstern

\let\aufgstar\aufgstern
\let\thmstar\thmstern
\let\thmconstar\thmconstern
\let\pfstar\pfstern
\let\behauptungstar\behauptungstern



\newtcolorbox{linebox}{colback=gray!70, colframe=gray!70, boxrule=0pt, left=0pt, right=0pt, top=0pt, bottom=0pt, width=\textwidth, height=0.6pt, arc=0.3pt, boxsep=0pt, before skip=1em, after skip=1em}




\newcommand*\circled[1]{
	\tikz[baseline = (char.base)]{
		\node[shape=circle,draw,inner sep=1pt] (char) {#1};}}
\newcommand\getcurrentref[1]{%
	\ifnumequal{\value{#1}}{0}
	{??}
	{\the\value{#1}}%
}
\newcommand{\getCurrentSectionNumber}{\getcurrentref{section}}
\newenvironment{myproof}[1][\proofname]{%
	\proof[\bfseries #1: ]%
}{\endproof}

\newcommand{\mclm}[2]{\begin{myclaim}[#1]#2\end{myclaim}}
\newenvironment{myclaim}[1][\claimname]{\proof[\bfseries #1: ]}{}

\newcounter{mylabelcounter}

\makeatletter
\newcommand{\setword}[2]{%
	\phantomsection{}
	#1\def\@currentlabel{\unexpanded{#1}}\label{#2}%
}
\makeatother


%%%%%%%%%%%%%%%%%%%%%%%%%%%%%%
% RANDOM ZEUG %
%%%%%%%%%%%%%%%%%%%%%%%%%%%%%%


\tikzset{
	symbol/.style={
			draw=none,
			every to/.append style={
					edge node={node [sloped, allow upside down, auto=false]{$#1$}}}
		}
}

\newsavebox\diffdbox{}
\newcommand{\slantedromand}{{\mathpalette\makesl{d}}}
\newcommand{\makesl}[2]{%
\begingroup
\sbox{\diffdbox}{$\mathsurround=0pt#1\mathrm{#2}$}%
\pdfsave{}
\pdfsetmatrix{1 0 0.2 1}%
\rlap{\usebox{\diffdbox}}%
\pdfrestore{}
\hskip\wd\diffdbox{}
\endgroup
}
\newcommand{\dd}[1][]{\ensuremath{\mathop{}\!\ifstrempty{#1}{%
\slantedromand\@ifnextchar^{\hspace{0.2ex}}{\hspace{0.1ex}}}%
{\slantedromand\hspace{0.2ex}^{#1}}}}
\ProvideDocumentCommand\dv{o m g}{%
  \ensuremath{%
    \IfValueTF{#3}{%
      \IfNoValueTF{#1}{%
        \frac{\dd #2}{\dd #3}%
      }{%
        \frac{\dd^{#1} #2}{\dd #3^{#1}}%
      }%
    }{%
      \IfNoValueTF{#1}{%
        \frac{\dd}{\dd #2}%
      }{%
        \frac{\dd^{#1}}{\dd #2^{#1}}%
      }%
    }%
  }%
}
\providecommand*{\pdv}[3][]{\frac{\partial^{#1}#2}{\partial#3^{#1}}}

% I prefer the slanted \leq
\let\oldleq\leq{} % save them in case they're every wanted
\let\oldgeq\geq{}
%\renewcommand{\leq}{\leqslant}
%\renewcommand{\geq}{\geqslant}

% % redefine matrix env to allow for alignment, use r as default
% \renewcommand*\env@matrix[1][r]{\hskip -\arraycolsep
%     \let\@ifnextchar\new@ifnextchar
%     \array{*\c@MaxMatrixCols #1}}


% Achtung Box
\newcommand{\achtung}[3]{
	\begin{center}
    \par\noindent
    \raisebox{0pt}[0.8\baselineskip][0.8\baselineskip]{\Huge\bfseries{}!}
    \hspace{#2}%
    \fbox{%
        \begin{minipage}{#1\linewidth}
			#3
        \end{minipage}%
    }%
    \hspace{#2}%
    \raisebox{0pt}[0.8\baselineskip][0.8\baselineskip]{\Huge\bfseries{}!} \\[7pt]
    \par
\end{center}
}




%%%%%%%%%%%%%%%%%%%%%%%%%%%%%%%%%%%%%%%%%%%
% TABLE OF CONTENTS
%%%%%%%%%%%%%%%%%%%%%%%%%%%%%%%%%%%%%%%%%%%

\usepackage{tikz}
\usetikzlibrary{patterns}
\usetikzlibrary{patterns.meta}
\usetikzlibrary{arrows}
\usepackage{titletoc}

\renewcommand{\chaptername}{Kapitel}
\newcommand{\pagename}{Seite}
\contentsmargin{0cm}
\titlecontents{part}[0pc]
{\color{doc}\huge\bfseries}
{}
{}
{\;\titlerule\;\large\bfseries \pagename\;\thecontentspage}
\titlecontents{chapter}[3pc]
{\addvspace{15pt}
\begin{tikzpicture}[remember picture, overlay]%
	\draw[fill=doc,draw=doc] (-7,-.2) rectangle (-0.3,.65);
	\pgftext[left,x=-3.5cm,y=0.2cm]{\color{white}\Large\;\;\bfseries \chaptername\;\thecontentslabel}
\end{tikzpicture}
\color{doc}\large\bfseries{}}%
{}
{}
{\;\titlerule\;\pagename\;\thecontentspage}
\titlecontents{section}[3pc]
{\addvspace{5pt}\color{doc}\large\bfseries}
{\color{doc}\thecontentslabel\quad}
{}
{\;\titlerule\;\pagename\;\thecontentspage}
[]
\titlecontents{subsection}[4pc]
{\addvspace{5pt}\color{doc}\large\bfseries}
{\color{doc}\thecontentslabel\quad}
{}
{\;\titlerule\;\pagename\;\thecontentspage}
[]
\setcounter{tocdepth}{3} % Zeigt bis zur Ebene "subsubsection"
\setcounter{secnumdepth}{3} % Nummeriert bis zur Ebene "subsubsection"
\titlecontents{subsubsection}[5pc]
{\addvspace{5pt}\color{doc}\large\bfseries}
{\color{doc}\thecontentslabel\quad}
{}
{\;\titlerule\;\pagename\;\thecontentspage}
[]
\titlecontents{lecture} % Das ist das richtige
[0pc]
{\centering\addvspace{5pt}\color{doc}\hfill}
{}
{}
{\hfill\;\thecontentspage}
\titlecontents{extras}[3pc]
{\addvspace{30pt}\color{white}%
\begin{tikzpicture}[remember picture, overlay]%
	\draw[fill=doc,draw=doc] (-7,-.2) rectangle (-0.3,.65);% -3.4 nach links reicht
	\pgftext[left,x=-3.5cm,y=0.2cm]{\color{white}\Large\;\;\bfseries\vphantom{bp}\;\thecontentslabel};%
	\node[star, star points=5, star point ratio=0.39, draw, fill=white, minimum size=5pt, rotate=180] at (-.8,0.22) {};%
\end{tikzpicture}
\color{white}\large\bfseries}%
{\color{white}}
{\color{white}}
{\color{doc}\;\titlerule\;\pagename\;\thecontentspage}




\makeatletter
\renewcommand{\tableofcontents}{%
\hypertarget{toc}{}
%\index{00@Inhaltsverzeichnis\hypertarget{indexstart}{}}
\chapter*{
	\vspace*{-20\p@}
	\begin{tikzpicture}[remember picture, overlay]
		\pgftext[right,x=14.9cm,y=0.2cm]{\color{doc}\Huge\sc\bfseries \contentsname}
		\draw[fill=doc,draw=doc] (\balkenx,-.75) rectangle (20,1);
		\clip (\balkenx,-.75) rectangle (20,1);
		\pgftext[right,x=14.9cm,y=0.2cm]{\color{white}\Huge\sc\bfseries \contentsname}
	\end{tikzpicture}
}%
\begingroup
\hypersetup{linkcolor=doc}% << Hier TOC-Linkfarbe setzen
\@starttoc{toc}
\endgroup
}
\makeatother



%%%%%%%%%%%%%%%%%%%%%%%%%%%%%%%%%%%%%%%%%%%
% ALIGN VSPACE %
%%%%%%%%%%%%%%%%%%%%%%%%%%%%%%%%%%%%%%%%%%%

\setlength{\jot}{5pt}


%%%%%%%%%%%%%%%%%%%%%%%%%%%%%%%%%%%%%%%%%%%
% SIMPLE REFERENCE MAKER WITH CORRECT COUNTER %
%%%%%%%%%%%%%%%%%%%%%%%%%%%%%%%%%%%%%%%%%%%

\newcommand{\preplabel}[1]{%
	\addtocounter{#1}{-1}%
	\refstepcounter{#1}%
	\tikz[remember picture,overlay]{%
	\node[anchor=south,yshift=2.5em] (#1) at (0, 0) {\phantomsection};
	}%
	\unskip\ignorespaces\unskip
}
\let\oldlabel\label{}
\renewcommand{\label}[1]{\unskip\oldlabel{#1}\ignorespaces} %\ignorespaces gegen Einrückung am Zeilenanfang


\newcommand{\oldpreplabel}[1]{
    \addtocounter{#1}{-1}
    \refstepcounter{#1}
}




% Markieren wie mit Textmarker
\newcommand{\highlight}[3]{\raisebox{-\dimexpr#1mm+1ex/2\relax}{\tcbox[colback=#2!40, colframe=#2!40, arc=2mm, boxrule=0mm, left=#1mm, right=#1mm, top=#1mm, bottom=#1mm]{#3}}} % #1 = 0.5 ist gut oder etwas kleiner

% In der Summe stehen die Grenzen immer drüber, auch bei anderen Sachen (kann mit \sum\nolimits_{} kurzzeitig überschrieben werden)
\everymath{\displaystyle}

% Fetter Text auch im Mathemodus
\let\oldtextbf\textbf
\renewcommand{\textbf}[1]{{\ifmmode%
\bm{#1}%
\else
\oldtextbf{\mathversion{bold}#1}%
\fi}}
% (Die geschweiften Klammern sind ganz wichtig, damit Latex in einer Matheumgebung an erster Stelle nicht denkt, das was jetzt kommt ist nicht im Mahtemodus)


% Das leere Zeichen, für die Turingmaschine
\let\oldtextvisiblespace\textvisiblespace{}
\renewcommand{\textvisiblespace}{\textsf{\oldtextvisiblespace}\hspace{0pt}}


% % \ldots oder \dotsc je nach Mathemodus oder nicht, denn \ldots im Mathemodus kann schlecht sein.
% \let\oldldots\ldots{}
% \renewcommand{\ldots}{%
%   \ifmmode%
%     \dotsc%
%   \else
%     \oldldots%
%   \fi
% }
\newcommand{\mydots}{\ensuremath{\dotsc}}
\renewcommand{\ldots}{\ensuremath{\dotsc}}




% Index am Ende
\let\oldprintindex\printindex
\renewcommand{\printindex}{
	\newpage
	% \begin{tikzpicture}[remember picture,overlay]
	%   \node (topanchor) at ([yshift=-1cm]current page.north) {\phantomsection};
	% \end{tikzpicture}
	\phantomsection{}
	\addcontentsline{toc}{extras}{Index}
	\hypertarget{indexstart}{}
	\oldprintindex{}
}

% Man kann auch mehrere Indexeinträge gleichzeitig anlegen
\let\oldindex\index{}
\renewcommand{\index}[1]{%
  \foreach\x in{#1}{\oldindex{\x}}\ignorespaces% %\ignorspaces sehr wichtig, damit keine Einrückung am Zeilenanfang entsteht
}

% Durchstreichen in Farbe, wenn man kürzen kann für Mathemodus und Textmodus
\newcommand{\ccancel}[2]{%
	\ifmmode%
	\tikz[baseline = (X.base)]{%
		\node[inner sep=0pt, outer sep=0pt] (X) {$#2$};
		\draw[#1, thick] (X.south west)-- (X.north east);}
	\else%
	\tikz[baseline = (X.base)]{%
		\node[inner sep=0pt, outer sep=0pt] (X) {#2};
		\draw[#1, thick] (X.south west)-- (X.north east);}
	\fi%
}

% Zusammenfassung mit multicols in Pistaziengrün
\definecolor{decentcolor}{HTML}{BBD686}
\newtcolorbox{decentbox}{
	before=\vspace{7pt},
	enhanced,
	breakable,
	boxsep=2pt,
	arc=12pt,
	colback=decentcolor!10,
	boxrule=2pt,
	colframe=decentcolor!25,
	valign=center,
	halign=center
}
\newcommand{\wichtigeDinge}[2]{
    \begin{decentbox}%
        \vphantom{Pp}#1%
    \end{decentbox}%
	\if\relax\detokenize{#2}\relax
		% Argument 2 ist leer – nichts tun
	\else
		\begin{multicols}{2}%
		#2%
		\end{multicols}%
	\fi
}

% Eine Leere Seite einfügen, wenn Seite mehr als 50% beschrieben
\makeatletter
\newcommand{\maybeAddEmptyPage}{%
  \par\vspace{0pt}% sichert vorherigen Absatz ab
  \begingroup
    \edef\currentlabel{maybeemptypage-\the\value{page}}%
    \zref@labelbyprops{\currentlabel}{abspage}%
    \ifdim\pagetotal{} > 0.5\pagegoal{}
      \clearpage
      \null\newpage
    \fi
  \endgroup
}
\makeatother

% Datum aus \lecturedivider in Counter speichern
\newcounter{mydate}
\makeatletter
\newcommand{\splitdate}[1]{%
	% Zerlegung: Teile #1 = DD.MM.YYYY in parsedday, parsedmonth, parsedyear
	\def\@splitdate##1.##2.##3\relax{%
		\def\parsedday{##1}%
		\def\parsedmonth{##2}%
		\def\parsedyear{##3}%
	}%
	\expandafter\@splitdate#1\relax
	% Nun parsedyear, parsedmonth, parsedday enthalten Strings wie "08", "04", "2025"
	% Setze mydate auf YYYYMMDD:
	\setcounter{mydate}{\numexpr\parsedyear*10000 + \parsedmonth*100 + \parsedday\relax}%
}
\makeatother

% include umdefinieren, sodass vielleicht eine leere Seite hinzugefügt wird, wenn die Vorlesung beim aktuellen war
\let\oldinclude\include
\renewcommand{\include}[1]{%
	\clearpage
	\begingroup
		\input{#1}%
		\ifnum\numexpr\value{mydate}\relax < \numexpr\year*10000+\month*100+\day\relax%
			\clearpage%
		\else%
			\maybeAddEmptyPage{}
			\clearpage
		\fi
	\endgroup
}

% Vorlesungstrenner
\newcounter{vorlesung}
\newcommand{\lecturedivider}[1]{
	\hypertarget{lecture.\thevorlesung}{}%
	\vspace{1em}
	\splitdate{#1}
	\noindent
	\begin{linebox}\end{linebox}

	\refstepcounter{vorlesung} % Zähler erhöhen
	% \addcontentsline{toc}{lecture}{\thevorlesung. Vorlesung vom #1} % Vorlesung im Inhaltsverzeichnis eintragen
	\addtocontents{toc}{\protect\contentsline{lecture}{\thevorlesung.\ Vorlesung vom #1}{\thepage}{lecture.\thevorlesung}} % Füge ins Inhaltsverzeichnis, aber nicht in die Metadaten hinzu
	\pdfbookmark[2]{\thevorlesung.\ Vorlesung vom #1}{bm:lecture.\thevorlesung} % Füge Vorlesung zu Metadaten hinzu untergeordnet

	\noindent\makebox[\textwidth][c]{\textbf{\thevorlesung. Vorlesung vom #1}} % Date centered
	\noindent
	\begin{linebox}\end{linebox}
	\vspace{1em}
}


% \newcommand{\lecturedivider}[1]{%
%   \refstepcounter{vorlesung}%
%   % --- normal ins Inhaltsverzeichnis (unter Kapitel einordnen) ---
%   \addcontentsline{toc}{section}{\thevorlesung. Vorlesung vom #1}%
%   % --- zusätzlich in separate Liste schreiben ---
%   \addcontentsline{lec}{lecture}{\thevorlesung. Vorlesung vom #1}%
%   % --- eigenes PDF-Lesezeichen erzeugen ---
%   \pdfbookmark[1]{\thevorlesung. Vorlesung vom #1}{lecture.\thevorlesung}%
%   % --- Überschrift im Text ---
%   \vspace{1em}%
%   \noindent\makebox[\textwidth][c]{\textbf{\thevorlesung. Vorlesung vom #1}}%
%   \vspace{1em}%
% }



%%%%%%%%%%%%%%%%%%%%%%%%%%%%%%%%%%%%%%%%%%%
% Einstellungen zu Countern
%%%%%%%%%%%%%%%%%%%%%%%%%%%%%%%%%%%%%%%%%%%

% globalcounter wird zu Beginn definiert
\newcounter{counter1}
\newcounter{counter2}
\newcounter{counter3}


% no counter mit separator sign=\hspace{-4pt}:
\newcommand{\countergroupA}{
	% Der globale Zähler ist standardmäßig ausgewählt ohne \countergroupA
	\renewtcbtheorem[number within=chapter, use counter=globalcounter]{Theorem}{Theorem}{Theorem style}{th}
	\renewtcbtheorem[number within=chapter, use counter=globalcounter]{Corollary}{Korollar}{Corollary style}{th}
	\renewtcbtheorem[number within=chapter, use counter=globalcounter]{mlemma}{Lemma}{mlemma style}{th}
	\renewtcbtheorem[number within=chapter, use counter=globalcounter]{fact}{Fakt}{fact style}{th}
	\renewtcbtheorem[number within=chapter, use counter=globalcounter]{notation}{Notation}{notation style}{th}
	\renewtcbtheorem[number within=chapter, use counter=globalcounter]{Prop}{Proposition}{Prop style}{th}
	\renewtcbtheorem[number within=chapter, use counter=globalcounter]{claim}{Satz}{claim style}{th}
	\renewtcbtheorem[number within=chapter, use counter=globalcounter]{Example}{Beispiel}{Example style}{ex}
	\renewtcbtheorem[number within=chapter, use counter=globalcounter]{Definition}{Definition}{Definition style}{def}
	\renewtcbtheorem[number within=chapter, use counter=globalcounter]{note}{Bemerkung}{note style}{def}
	\renewtcbtheorem[number within=chapter, use counter=globalcounter]{Folgerung}{Folgerung}{Folgerung style}{folg}
	\renewtcbtheorem[number within=chapter, use counter=aufg]{Exercise}{Aufgabe}{Exercise style}{aufg}
	\renewcommand{\thealgocf}{\thechapter.\arabic{algocf}}
	\counterwithin{algocf}{chapter}
}
\newcommand{\countergroupB}{
	% Wer alles einzeln gezählt möchte
	\renewtcbtheorem[number within=chapter, use counter=thm]{Theorem}{Theorem}{Theorem style}{th}
	\renewtcbtheorem[number within=chapter, use counter=cor]{Corollary}{Korollar}{Corollary style}{th}
	\renewtcbtheorem[number within=chapter, use counter=lemma]{mlemma}{Lemma}{mlemma style}{th}
	\renewtcbtheorem[number within=chapter, use counter=fact]{fact}{Fakt}{fact style}{th}
	\renewtcbtheorem[number within=chapter, use counter=notation]{notation}{Notation}{notation style}{th}
	\renewtcbtheorem[number within=chapter, use counter=prop]{Prop}{Proposition}{Prop style}{th}
	\renewtcbtheorem[number within=chapter, use counter=clm]{claim}{Satz}{claim style}{th}
	\renewtcbtheorem[number within=chapter, use counter=ex]{Example}{Beispiel}{Example style}{ex}
	\renewtcbtheorem[number within=chapter, use counter=dfn]{Definition}{Definition}{Definition style}{def}
	\renewtcbtheorem[number within=chapter, use counter=nt]{note}{Bemerkung}{note style}{def}
}
\newcommand{\countergroupC}{
	% Orientiert sich an Koch (leider nicht immer möglich)
	\renewtcbtheorem[number within=chapter, use counter=counter1]{Theorem}{Theorem}{Theorem style}{th}
	\renewtcbtheorem[number within=chapter, use counter=counter1]{Corollary}{Korollar}{Corollary style}{th}
	\renewtcbtheorem[number within=chapter, use counter=counter1]{mlemma}{Lemma}{mlemma style}{th}
	\renewtcbtheorem[number within=chapter, use counter=counter1]{fact}{Fakt}{fact style}{th}
	\renewtcbtheorem[number within=chapter, use counter=counter1]{notation}{Notation}{notation style}{th}
	\renewtcbtheorem[number within=chapter, use counter=counter1]{Prop}{Proposition}{Prop style}{th}
	\renewtcbtheorem[number within=chapter, use counter=counter1]{claim}{Satz}{claim style}{th}
	\renewtcbtheorem[number within=chapter, use counter=ex]{Example}{Beispiel}{Example style}{ex}
	\renewtcbtheorem[number within=chapter, use counter=dfn]{Definition}{Definition}{Definition style}{def}
	\renewtcbtheorem[number within=chapter, use counter=counter1]{note}{Bemerkung}{note style}{def}
}
\newcommand{\countergroupD}{
	% Ähnliche Dinge zusammengefasst
	\renewtcbtheorem[number within=chapter, use counter=clm]{Theorem}{Theorem}{Theorem style}{th}
	\renewtcbtheorem[number within=chapter, use counter=lemma]{Corollary}{Korollar}{Corollary style}{th}
	\renewtcbtheorem[number within=chapter, use counter=lemma]{mlemma}{Lemma}{mlemma style}{th}
	\renewtcbtheorem[number within=chapter, use counter=nt]{fact}{Fakt}{fact style}{th}
	\renewtcbtheorem[number within=chapter, use counter=ex]{notation}{Notation}{notation style}{th}
	\renewtcbtheorem[number within=chapter, use counter=lemma]{Prop}{Proposition}{Prop style}{th}
	\renewtcbtheorem[number within=chapter, use counter=clm]{claim}{Satz}{claim style}{th}
	\renewtcbtheorem[number within=chapter, use counter=ex]{Example}{Beispiel}{Example style}{ex}
	\renewtcbtheorem[number within=chapter, use counter=dfn]{Definition}{Definition}{Definition style}{def}
	\renewtcbtheorem[number within=chapter, use counter=nt]{note}{Bemerkung}{note style}{def}
}
\newcommand{\countergroupEDM}{
	% Ähnliche Dinge zusammengefasst
	\renewtcbtheorem[use counter=globalcounter]{Theorem}{Theorem}{Theorem style}{th}
	\renewtcbtheorem[use counter=globalcounter]{Corollary}{Korollar}{Corollary style}{th}
	\renewtcbtheorem[use counter=globalcounter]{mlemma}{Lemma}{mlemma style}{th}
	\renewtcbtheorem[use counter=globalcounter]{fact}{Fakt}{fact style}{th}
	\renewtcbtheorem[use counter=globalcounter]{notation}{Notation}{notation style}{th}
	\renewtcbtheorem[use counter=globalcounter]{Prop}{Proposition}{Prop style}{th}
	\renewtcbtheorem[use counter=globalcounter]{claim}{Satz}{claim style}{th}
	\renewtcbtheorem[use counter=globalcounter]{Example}{Beispiel}{Example style}{ex}
	\renewtcbtheorem[no counter]{Definition}{Definition}{Definition style, theorem name}{def}
	\renewtcbtheorem[no counter]{note}{Bemerkung}{note style, theorem name}{def}
}
\newcommand{\countergroupGruNum}{
	% Der globale Zähler ist standardmäßig ausgewählt ohne \countergroupA
	\renewtcbtheorem[number within=chapter, use counter=globalcounter]{Theorem}{Theorem}{Theorem style}{th}
	\renewtcbtheorem[number within=chapter, use counter=globalcounter]{Corollary}{Korollar}{Corollary style}{th}
	\renewtcbtheorem[number within=chapter, use counter=globalcounter]{mlemma}{Lemma}{mlemma style}{th}
	\renewtcbtheorem[number within=chapter, use counter=globalcounter]{fact}{Fakt}{fact style}{th}
	\renewtcbtheorem[no counter]{notation}{Notation}{notation style}{th}
	\renewtcbtheorem[number within=chapter, use counter=globalcounter]{Prop}{Proposition}{Prop style}{th}
	\renewtcbtheorem[number within=chapter, use counter=globalcounter]{claim}{Satz}{claim style}{th}
	\renewtcbtheorem[number within=chapter, use counter=globalcounter]{Example}{Beispiel}{Example style}{ex}
	\renewtcbtheorem[number within=chapter, use counter=globalcounter]{Definition}{Definition}{Definition style}{def}
	\renewtcbtheorem[number within=chapter, use counter=globalcounter]{note}{Bemerkung}{note style}{def}
	\numberwithin{algocf}{chapter}
}
\newcommand{\countergroupTI}{
	% Der globale Zähler ist standardmäßig ausgewählt ohne \countergroupA
	\renewtcbtheorem[number within=chapter, use counter=globalcounter]{Theorem}{Theorem}{Theorem style}{th}
	\renewtcbtheorem[number within=chapter, use counter=globalcounter]{Corollary}{Korollar}{Corollary style}{th}
	\renewtcbtheorem[number within=chapter, use counter=globalcounter]{mlemma}{Lemma}{mlemma style}{th}
	\renewtcbtheorem[number within=chapter, use counter=globalcounter]{fact}{Fakt}{fact style}{th}
	\renewtcbtheorem[number within=chapter, use counter=globalcounter]{notation}{Notation}{notation style}{th}
	\renewtcbtheorem[number within=chapter, use counter=globalcounter]{Prop}{Proposition}{Prop style}{th}
	\renewtcbtheorem[number within=chapter, use counter=globalcounter]{claim}{Satz}{claim style}{th}
	\renewtcbtheorem[number within=chapter, use counter=globalcounter]{Example}{Beispiel}{Example style}{ex}
	\renewtcbtheorem[number within=chapter, use counter=globalcounter]{Definition}{Definition}{Definition style}{def}
	\renewtcbtheorem[number within=chapter, use counter=globalcounter]{note}{Bemerkung}{note style}{def}
}
\newcommand{\countergroupAnalysis}{
	% Der globale Zähler ist standardmäßig ausgewählt ohne \countergroupA
	\renewtcbtheorem[number within=chapter, use counter=globalcounter]{Theorem}{Theorem}{Theorem style}{th}
	\renewtcbtheorem[number within=chapter, use counter=globalcounter]{Corollary}{Korollar}{Corollary style}{th}
	\renewtcbtheorem[number within=chapter, use counter=globalcounter]{mlemma}{Lemma}{mlemma style}{th}
	\renewtcbtheorem[number within=chapter, use counter=globalcounter]{fact}{Fakt}{fact style}{th}
	\renewtcbtheorem[number within=chapter, use counter=globalcounter]{notation}{Notation}{notation style}{th}
	\renewtcbtheorem[number within=chapter, use counter=globalcounter]{Prop}{Proposition}{Prop style}{th}
	\renewtcbtheorem[number within=chapter, use counter=globalcounter]{claim}{Satz}{claim style}{th}
	\renewtcbtheorem[number within=chapter, use counter=ex]{Example}{Beispiel}{Example style}{ex}
	\renewtcbtheorem[number within=chapter, use counter=dfn]{Definition}{Definition}{Definition style}{def}
	\renewtcbtheorem[no counter]{note}{Bemerkung}{note style, theorem name}{def}
}
\newcommand{\countergroupLernEDM}{
	\counterwithout{globalcounter}{chapter}
	% Der globale Zähler ist standardmäßig ausgewählt ohne \countergroupA
	\renewtcbtheorem[no counter]{Theorem}{Theorem}{Theorem style}{th}
	\renewtcbtheorem[use counter=globalcounter]{Corollary}{Korollar}{Corollary style}{th}
	\renewtcbtheorem[use counter=globalcounter]{mlemma}{Lemma}{mlemma style}{th}
	\renewtcbtheorem[no counter]{fact}{Fakt}{fact style}{th}
	\renewtcbtheorem[no counter]{notation}{Notation}{notation style}{th}
	\renewtcbtheorem[use counter=globalcounter]{Prop}{Proposition}{Prop style}{th}
	\renewtcbtheorem[use counter=globalcounter]{claim}{Satz}{claim style}{th}
	\renewtcbtheorem[no counter]{Example}{Beispiel}{Example style}{ex}
	\renewtcbtheorem[use counter=counter1]{Definition}{Definition}{Definition style}{def}
	\renewtcbtheorem[no counter]{note}{Bemerkung}{note style}{def}
}
\newcommand{\countergroupLernAna}{
	\counterwithout{globalcounter}{chapter}
	% Der globale Zähler ist standardmäßig ausgewählt ohne \countergroupA
	\renewtcbtheorem[no counter]{Theorem}{Theorem}{Theorem style}{th}
	\renewtcbtheorem[no counter]{Corollary}{Korollar}{Corollary style}{th}
	\renewtcbtheorem[no counter]{mlemma}{Lemma}{mlemma style}{th}
	\renewtcbtheorem[no counter]{fact}{Fakt}{fact style}{th}
	\renewtcbtheorem[no counter]{notation}{Notation}{notation style}{th}
	\renewtcbtheorem[no counter]{Prop}{Proposition}{Prop style}{th}
	\renewtcbtheorem[no counter]{claim}{Satz}{claim style}{th}
	\renewtcbtheorem[no counter]{Example}{Beispiel}{Example style}{ex}
	\renewtcbtheorem[no counter]{Definition}{Definition}{Definition style}{def}
	\renewtcbtheorem[no counter]{note}{Bemerkung}{note style}{def}
}
%\renewtcbtheorem[no counter]{Definition}{Definition}{Definition style, separator sign=\hspace{-.3em}:}{def}
\newcommand{\countergroupLogik}{
	% Der globale Zähler ist standardmäßig ausgewählt ohne \countergroupA
	\renewtcbtheorem[number within=section, use counter=globalcounter]{Theorem}{Theorem}{Theorem style}{th}
	\renewtcbtheorem[number within=section, use counter=globalcounter]{Corollary}{Korollar}{Corollary style}{th}
	\renewtcbtheorem[number within=section, use counter=globalcounter]{mlemma}{Lemma}{mlemma style}{th}
	\renewtcbtheorem[number within=section, use counter=globalcounter]{fact}{Fakt}{fact style}{th}
	\renewtcbtheorem[no counter]{notation}{Notation}{notation style}{th}
	\renewtcbtheorem[number within=section, use counter=globalcounter]{Prop}{Proposition}{Prop style}{th}
	\renewtcbtheorem[number within=section, use counter=globalcounter]{claim}{Satz}{claim style}{th}
	\renewtcbtheorem[number within=section, use counter=globalcounter]{Example}{Beispiel}{Example style}{ex}
	\renewtcbtheorem[number within=section, use counter=globalcounter]{Definition}{Definition}{Definition style}{def}
	\renewtcbtheorem[number within=section, use counter=globalcounter]{note}{Bemerkung}{note style}{def}
	\renewtcbtheorem[number within=chapter, use counter=aufg]{Exercise}{Aufgabe}{Exercise style}{aufg}
	\renewtcbtheorem[number within=section, use counter=globalcounter]{Folgerung}{Folgerung}{Folgerung style}{folg}
	% \renewcommand{\thealgocf}{\thechapter.\arabic{algocf}}
	% \counterwithin{algocf}{chapter}
	\renewcommand{\theglobalcounter}{\thesection.\arabic{globalcounter}}
	\counterwithin{globalcounter}{section}
}
\newcommand{\countergroupGeoTopo}{
	% Der globale Zähler ist standardmäßig ausgewählt ohne \countergroupA
	\renewtcbtheorem[number within=chapter, use counter=globalcounter]{Theorem}{Theorem}{Theorem style}{th}
	\renewtcbtheorem[use counter=cor]{Corollary}{Korollar}{Corollary style}{cor}
	\renewtcbtheorem[number within=chapter, use counter=globalcounter]{mlemma}{Lemma}{mlemma style}{th}
	\renewtcbtheorem[no counter]{fact}{Fakt}{fact style}{th}
	\renewtcbtheorem[no counter]{question}{Frage}{question style}{def}
	\renewtcbtheorem[no counter]{notation}{Notation}{notation style}{th}
	\renewtcbtheorem[number within=chapter, use counter=globalcounter]{Prop}{Proposition}{Prop style}{th}
	\renewtcbtheorem[number within=chapter, use counter=globalcounter]{claim}{Satz}{claim style}{th}
	\renewtcbtheorem[no counter]{Example}{Beispiel}{Example style}{ex}
	\renewtcbtheorem[number within=chapter, use counter=globalcounter]{Definition}{Definition}{Definition style}{def}
	\renewtcbtheorem[no counter]{note}{Bemerkung}{note style}{def}
	\numberwithin{algocf}{chapter}
}

% Definition ohne counter:  \renewtcbtheorem[no counter]{Definition}{Definition}{Definition style, separator sign=\hspace{-.3em}:}{def}
